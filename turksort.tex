\documentclass{article}

\usepackage{listings}

\title{A Novel Type-Independent Sorting Algorithm}
\author{Cole Kurashige \\ Harvey Mudd College}
\date{2020 \\ March}

\begin{document}

\section{Abstract}
This section is abstract.

\section{Background}
Sorting is generally thought to be a mostly-solved problem. Lower bounds for
its time and space complexity have been long-established for comparison-based
sorting algorithms. Little has been done, however, to examine what it means to
make a comparison.

In statically-typed programming languages like Java or C++, comparisons such as
greater than (\texttt{>}) or less than (\texttt{<}) are often restricted to
operating on two elements of the same type. These languages consider it a
compliation error to compare elements of differing types.

Dynamically-typed programming languages like Python do not consider it a
compilation error; however, it is a runtime error to make these

\begin{lstlisting}[caption = Comparisons in Python 3.7.6]
>>> 1 > "-1"
Traceback (most recent call last):
  File "<stdin>", line 1, in <module>
TypeError: '>' not supported between instances of 'int' and 'str'
\end{lstlisting}

\end{document}